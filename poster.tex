%%%%%%%%%%%%%%%%%%%%%%%%%%%%%%%%%%%%%%%%%
% Dreuw & Deselaer's Poster
% LaTeX Template
% Version 1.0 (11/04/13)
%
% Created by:
% Philippe Dreuw and Thomas Deselaers
% http://www-i6.informatik.rwth-aachen.de/~dreuw/latexbeamerposter.php
%
% This template has been downloaded from:
% http://www.LaTeXTemplates.com
%
% License:
% CC BY-NC-SA 3.0 (http://creativecommons.org/licenses/by-nc-sa/3.0/)
%
%%%%%%%%%%%%%%%%%%%%%%%%%%%%%%%%%%%%%%%%%

%----------------------------------------------------------------------------------------
%	PACKAGES AND OTHER DOCUMENT CONFIGURATIONS
%----------------------------------------------------------------------------------------

\documentclass[final,hyperref={pdfpagelabels=false}]{beamer}

\usepackage[orientation=landscape,size=a0,scale=1.4]{beamerposter} % Use the beamerposter package for laying out the poster with a portrait orientation and an a0 paper size

\usetheme{I6pd2} % Use the I6pd2 theme supplied with this template

\usepackage[english]{babel} % English language/hyphenation

\usepackage{amsmath,amsthm,amssymb,latexsym} % For including math equations, theorems, symbols, etc

%\usepackage{times}\usefonttheme{professionalfonts}  % Uncomment to use Times as the main font
%\usefonttheme[onlymath]{serif} % Uncomment to use a Serif font within math environments

\boldmath % Use bold for everything within the math environment

\usepackage{booktabs} % Top and bottom rules for tables

\graphicspath{{figures/}} % Location of the graphics files

\usecaptiontemplate{\small\structure{\insertcaptionname~\insertcaptionnumber: }\insertcaption} % A fix for figure numbering


% Additional packages
\usepackage[misc]{ifsym}

% commands
\newcommand{\todo}[1]{\textcolor{red}{\{\textbf{#1}\}}}

%----------------------------------------------------------------------------------------
%	TITLE SECTION 
%----------------------------------------------------------------------------------------

\title{\huge The Intrinsic Manifolds of Radiological Images and their Role in Deep Learning} % Poster title

\author{Nicholas Konz\textsuperscript{1}\and Hanxue Gu\textsuperscript{1} \and Haoyu Dong\textsuperscript{2} \and Maciej Mazurowski\textsuperscript{1,2,3,4} } % Author(s)

\institute{\textsuperscript{1}Department of Electrical and Computer Engineering, \textsuperscript{2}Department of Radiology, \textsuperscript{3}Department of Computer Science, \textsuperscript{4}Department of Biostatistics \& Bioinformatics,\\Duke University, Durham, North Carolina, USA} % Institution(s)

%----------------------------------------------------------------------------------------
%	FOOTER TEXT
%----------------------------------------------------------------------------------------

\newcommand{\leftfoot}{http://www.LaTeXTemplates.com} % Left footer text

\newcommand{\rightfoot}{} % Right footer text

%----------------------------------------------------------------------------------------

\begin{document}

\addtobeamertemplate{block end}{}{\vspace*{2ex}} % White space under blocks

\begin{frame}[t] % The whole poster is enclosed in one beamer frame

\begin{columns}[t] % The whole poster consists of two major columns, each of which can be subdivided further with another \begin{columns} block - the [t] argument aligns each column's content to the top

\begin{column}{.02\textwidth}\end{column} % Empty spacer column

\begin{column}{.3\textwidth} % The first column

\begin{block}{Introduction}

\begin{columns} % Subdivide the first main column
\begin{column}{.54\textwidth} % The first subdivided column within the first main column
\begin{itemize}
\item \textbf{The Manifold Hypothesis (MH)}: \textit{High dimensional data can be well described by a much smaller number of degrees of freedom, or} \textbf{intrinsic dimensions}.
    \begin{itemize}
    \item The intrinsic dimension of a dataset estimates its information content. 
    \item Neural networks can learn to convert raw data to abstract, informative features that are \textit{intrinsic} to the dataset \cite{ansuini2019intrinsic}.
    \end{itemize}

\end{itemize}
\end{column}

\begin{column}{.43\textwidth} % The second subdivided column within the first main column
\centering
\begin{figure}
    \includegraphics[width=0.95\linewidth]{buchanan_etal_fig1.pdf}
     \caption{\,Visualization of intrinsic low-dimensional image manifolds from \cite{buchanan2021deep}.}
\end{figure}
\end{column}
\end{columns} % End of the subdivision
\begin{itemize}
\item \textbf{Why study the intrinsic dimension of medical images?}
    \begin{itemize}
        \item Medical vs. natural image datasets: different relevant semantics, yet we lack understanding of how networks learn differently between the two domains.
        \item Due to the MH, understanding the intrinsic structure of medical image datasets is key to analyzing how networks learn from them.
        \item We can answer questions such as ``how does the intrinsic dimension of a dataset affect how challenging it is for a network to learn from it?''
    \end{itemize}
\end{itemize}

\end{block}

\begin{block}{Objectives}

\begin{enumerate}
\item Estimate the intrinsic dimensions of common radiology datasets, and compare to natural image datasets.
\item Evaluate the relationship of dataset intrinsic dimension with network generalization ability; comparing within and between the domains of radiological and natural images.
\end{enumerate}

\end{block}

\begin{block}{Estimating the Intrinsic Dimension of Image Manifolds}
\begin{itemize}
    \item The MH says that our $d$-dimensional data lies on a manifold $\mathcal{M}\subseteq \mathbb{R}^d$ such that $\mathrm{dim}{\mathcal{M}}=m\ll d$.
\item We can estimate $m$ via \textbf{maximum likelihood}:
    \begin{itemize}
        \item assume that manifold volume scales exponentially with $m$ as we move away from a point. 
        \item model volume with $k$-nearest neighbor distance $T_k$.
        \item Model dataset as being sampled from a Poisson Process, and find $m$ via MLE:
        \begin{equation*}
            \label{eq:id_mle}
                \hat{m}=\left[\frac{1}{N(k-1)} \sum_{i=1}^{N} \sum_{j=1}^{k-1} \log \frac{T_{k}\left(x_{i}\right)}{T_{j}\left(x_{i}\right)}\right]^{-1}
        \end{equation*}
    \end{itemize}

\end{itemize}
\end{block}

%----------------------------------------------------------------------------------------

\end{column} % End of the first column
\begin{column}{.03\textwidth}\end{column} % Empty spacer column

 
\begin{column}{.3\textwidth} % The second column

\begin{block}{Datasets and Experimental Settings}

\begin{figure}
    \includegraphics[width=0.95\linewidth]{frompaper/data_eg_1row.pdf}
     \caption{\,Samples from our seven evaluated datasets.}
\end{figure}


\begin{itemize}
\item We analyze seven common radiology datasets from different modalities.
\item \textbf{ID estimation:}
\begin{itemize}
    \item For each dataset, we use a sample of $7500$ images, evenly class-balanced according to our chosen binary classification task for each.
\end{itemize}
\item \textbf{ID and Generalization Ability}:
\begin{itemize}
    \item We train a classification network for each dataset, and test on 750 unseen data points.
    \item We evaluate many training set sizes, neural network models, and task choices. All networks are maximally fit to the training set.
\end{itemize}
\end{itemize}

\end{block}

\begin{block}{Result 1: Radiological vs. Natural Image Intrinsic Dimension}
\begin{figure}
    \includegraphics[width=0.95\linewidth]{frompaper/modified/ID.pdf}
     \caption{\,Intrinsic dimension of \textcolor{paperblue}{radiological} and \textcolor{paperorange}{natural} \cite{pope2021intrinsic} image datasets.}
\end{figure}

\begin{itemize}
    \item \textbf{Findings:}
    \begin{itemize}
        \item \textbf{Radiological image datasets tend to have lower intrinsic dimension (ID) than natural image datasets!}
        \item ID $\ll$ extrinsic dimension (ED/num. of pixels).
        \begin{itemize}
            \item Intuitively, modifying ED (resizing images) didn't affect ID.
        \end{itemize}
    \end{itemize}
\end{itemize}

\end{block}

%----------------------------------------------------------------------------------------

\end{column} % End of the second column

\begin{column}{.03\textwidth}\end{column} % Empty spacer column
 
\begin{column}{.3\textwidth} % The third column


\begin{block}{Main Result: Intrinsic Dimension and Generalization Ability}

\begin{figure}
    \label{fig:linear}
    \includegraphics[width=0.95\linewidth]{frompaper/main_fig_multi_0.pdf}
     \caption{\,Linearity of model generalization ability with respect to dataset intrinsic dimension, for \textcolor{paperblue}{radiological} and \textcolor{paperorange}{natural} image datasets ($N_\text{train}=2000$ on ResNet-18).}
\end{figure}

\begin{columns}
\begin{column}{.54\textwidth} % The first subdivided column within the first main column
    \begin{itemize}
    \item Vestibulum nisl, quis euismod velit eros in ligula.
    \begin{itemize}
    \item Cras rhoncus quam et augue convallis in elementum urna tincidunt.
    \end{itemize}
    \item Proin ut vestibulum augue.
    \begin{itemize}
    \item Donec dapibus sagittis neque eu ultrices.
    \end{itemize}
    \end{itemize}
    \end{column}
    
    \begin{column}{.43\textwidth} % The second subdivided column within the first main column
    \centering
    \begin{figure}
        \includegraphics[width=0.95\linewidth]{frompaper/modified/main_fig_multi.pdf}
        \caption{\,Fig. \ref{fig:linear}, including training set size.}
    \end{figure}
    \end{column}
\end{columns}
\todo{numerical evidence showing how these relationships didn't change with model, dataset size, task, etc (see paper). Also additional results from "Contributions" in paper.}


\end{block}

\begin{block}{Open Questions/Future Work}

\begin{itemize}
\item Investigate theoretical reasons for linearity and difference in sharpness of correlation between the two domains
\item Determine further uses of ID estimate for predictions, experiments etc.
\end{itemize}

\end{block}


\begin{block}{References}
        
% \nocite{*} % Insert publications even if they are not cited in the poster
\small{\bibliographystyle{unsrt}
\bibliography{mainbib}}

\end{block}

\setbeamercolor{block title}{fg=black,bg=orange!70} % Change the block title color

\begin{block}{Contact Information}

\begin{itemize}
\item Web: \todo{lab website}
\item Email: \href{mailto:nicholas.konz@duke.edu}{nicholas.konz@duke.edu}
\end{itemize}

\end{block}
\end{column} % End of the third column


\begin{column}{.015\textwidth}\end{column} % Empty spacer column

\end{columns} % End of all the columns in the poster

\end{frame} % End of the enclosing frame

\end{document}